\subsection{HTML5}

Firefox lanserte 30. juni den f�rste nettleseren med st�tte for HTML5 video-taggen\footnoter{\url{http://www.engadget.com/2009/06/30/firefox-3-5-arrives/}}. Siden den tid har de fleste st�rste nettlesere kommet med en generell st�tte, med unntak av den st�rte, Internet Explorer, som vil gi st�tte med neste utgivelse, versjon 9. Noen lanseringsdato er forel�pig ikke fastsatt.

Det er vedvarende ingen enighet om hva slags format som skal brukes p� selve transportstr�mme, hverken innkapsling, eller formater innad i innkapslingen. Safari og den kommende utgaven av Internet Explorer, st�tter utelukkende innkapslingsformatet MP4, med h.264 video- og AAC som lydkodek. Firefox og Opera st�tter kun Ogg som innkapslingsformat, med Theora som video- og Vorbis som lydkodek. Chrome p� den andre siden st�tter begge formater.

I tillegg til utbredelsen av Ogg- og MP4-innkapslingsformatet, er det kommet et tredje alternativ til de to eksistende formatene. Formatet er utviklet av Google og g�r under navnet WebM\footnoter{\url{http://www.webmproject.org/}}. Formatet baserer seg p� VP8 som video- og Vorbis som lydkodek, hvor disse kodekene er faste. WebM er tilgjengeliggjort under en �pen lisens som tillater fritt kommerielt bruk. En mer grundig gjennomgang av WebM gj�res i et underkapittel funnet lenger ned.\\\\
%
\begin{tabular}{| c | l | l | l | l | l | l | l |}
  \hline
  \textbf{Format} & \textbf{IE} & \textbf{Firefox} & \textbf{Safari} & \textbf{Chrome} & \textbf{Opera} & \textbf{iPhone} & \textbf{Android} \\ \hline
  HTML5 video & \parbox[c]{1em}{\includegraphics[width=3mm]{img/icon-ball-blue.png}}\scriptsize{v9.0+} & \parbox[c]{1em}{\includegraphics[width=3mm]{img/icon-ball-green.png}}\scriptsize{v3.5+} & \parbox[c]{1em}{\includegraphics[width=3mm]{img/icon-ball-green.png}}\scriptsize{v3.0+} & \parbox[c]{1em}{\includegraphics[width=3mm]{img/icon-ball-green.png}} & \parbox[c]{1em}{\includegraphics[width=3mm]{img/icon-ball-green.png}} & \parbox[c]{1em}{\includegraphics[width=3mm]{img/icon-ball-green.png}} & \parbox[c]{1em}{\includegraphics[width=3mm]{img/icon-ball-green.png}} \\ \hline

  Ogg\scriptsize{(Theora+Vorbis)}  & \parbox[c]{1em}{\includegraphics[width=3mm]{img/icon-ball-red.png}} & \parbox[c]{1em}{\includegraphics[width=3mm]{img/icon-ball-green.png}}\scriptsize{v3.5+} & \parbox[c]{1em}{\includegraphics[width=3mm]{img/icon-ball-red.png}} & \parbox[c]{1em}{\includegraphics[width=3mm]{img/icon-ball-green.png}}\scriptsize{v5.0+} & \parbox[c]{1em}{\includegraphics[width=3mm]{img/icon-ball-green.png}}\scriptsize{v10.5+} & \parbox[c]{1em}{\includegraphics[width=3mm]{img/icon-ball-red.png}} & \parbox[c]{1em}{\includegraphics[width=3mm]{img/icon-ball-red.png}} \\ \hline

  MP4\scriptsize{(h.262+AAC)}  & \parbox[c]{1em}{\includegraphics[width=3mm]{img/icon-ball-red.png}} & \parbox[c]{1em}{\includegraphics[width=3mm]{img/icon-ball-red.png}} & \parbox[c]{1em}{\includegraphics[width=3mm]{img/icon-ball-green.png}}\scriptsize{v3.0+} & \parbox[c]{1em}{\includegraphics[width=3mm]{img/icon-ball-green.png}}\scriptsize{v5.0+} & \parbox[c]{1em}{\includegraphics[width=3mm]{img/icon-ball-red.png}} & \parbox[c]{1em}{\includegraphics[width=3mm]{img/icon-ball-green.png}}\scriptsize{v3.0+} & \parbox[c]{1em}{\includegraphics[width=3mm]{img/icon-ball-green.png}}\scriptsize{v2.0+} \\ \hline

  WebM\scriptsize{(VP8+Vorbis)}  & \parbox[c]{1em}{\includegraphics[width=3mm]{img/icon-ball-blue.png}}\scriptsize{v9.0+} \footmark & \parbox[c]{1em}{\includegraphics[width=3mm]{img/icon-ball-blue.png}}\scriptsize{v4.0+} & \parbox[c]{1em}{\includegraphics[width=3mm]{img/icon-ball-red.png}} \footmark & \parbox[c]{1em}{\includegraphics[width=3mm]{img/icon-ball-blue.png}}\scriptsize{v6.0+} & \parbox[c]{1em}{\includegraphics[width=3mm]{img/icon-ball-blue.png}}\scriptsize{v10.0+} & \parbox[c]{1em}{\includegraphics[width=3mm]{img/icon-ball-red.png}} & \parbox[c]{1em}{\includegraphics[width=3mm]{img/icon-ball-yellow.png}} \\ \hline
\end{tabular}

\foottext{brukeren m� installere en plugin}
\foottext{brukeren m� installere en 3.parts plugin for QuickTime}

%
\begin{tabular}{ c l }
  \parbox[c]{1em}{\includegraphics[width=3mm]{img/icon-ball-green.png}} & st�ttet i siste utgivelse \\
  \parbox[c]{1em}{\includegraphics[width=3mm]{img/icon-ball-blue.png}} & st�tte kommer i gitt fremtidig utgivelse \\
  \parbox[c]{1em}{\includegraphics[width=3mm]{img/icon-ball-yellow.png}} & st�tte kommer i ubestemt fremtidig utgivelse \\
  \parbox[c]{1em}{\includegraphics[width=3mm]{img/icon-ball-red.png}} & ingen st�tte, hverken n� eller i kjent fremtid \\
\end{tabular}

\subsection{WebM}

19. mai, 2010 lanserte Google det �pne innkapslingsformatet WebM, sammen med videoformatet VP8. Innkapslingen bruker Vorbis for lyd og VP8 for video, dermed er alle ledd basert p� �pne standarder.

Flere fordeler ligger til grunne for at dette formatet kan bli sv�rt utbredt:

\begin{itemize}
  \item formatet er �pent
  \item formatene brukt for video- og lydstr�mmer er strengt definerte
  \item mange store akt�rer stiller seg bak formatet
\end{itemize}
%
St�tte er allerede inkorporert i utviklerutgaver av flere store nettlesere, og flere er i vente.\\\\
%
\begin{tabular}{| c | l | l | l | l | l | l | l |}
  \hline
  \textbf{Format} & \textbf{IE} & \textbf{Firefox} & \textbf{Safari} & \textbf{Chrome} & \textbf{Opera} & \textbf{iPhone} & \textbf{Android} \\ \hline
  WebM\scriptsize{(VP8+Vorbis)}  & \parbox[c]{1em}{\includegraphics[width=3mm]{img/icon-ball-blue.png}}\scriptsize{v9.0+} \footmark & \parbox[c]{1em}{\includegraphics[width=3mm]{img/icon-ball-blue.png}}\scriptsize{v4.0+} & \parbox[c]{1em}{\includegraphics[width=3mm]{img/icon-ball-red.png}} & \parbox[c]{1em}{\includegraphics[width=3mm]{img/icon-ball-blue.png}}\scriptsize{v6.0+} & \parbox[c]{1em}{\includegraphics[width=3mm]{img/icon-ball-blue.png}}\scriptsize{v10.0+} & \parbox[c]{1em}{\includegraphics[width=3mm]{img/icon-ball-red.png}} & \parbox[c]{1em}{\includegraphics[width=3mm]{img/icon-ball-yellow.png}} \\ \hline
\end{tabular}

\foottext{brukeren m� installere en plugin}
\foottext{brukeren m� installere en 3.parts plugin for QuickTime}

\subsection{h.264 vs. Ogg vs. WebM}

\subsection{Underteksting}
