\documentclass[a4paper,norsk]{article}
\usepackage{graphicx}
\usepackage{palatino}
\usepackage[latin1]{inputenc}
\usepackage{hyperref}
\usepackage{fmtcount}
\usepackage{verbatim}

\title{Det siste �rs teknologiutvikling}
\author{}
\date{25. juni 2010}

\begin{document}
\maketitle

\section{Introduksjon}

\section{Standarder}

\subsection{HTML5}

Firefox lanserte 30. juni den f�rste nettleseren med st�tte for HTML5 video-taggen\footnote{\url{http://www.engadget.com/2009/06/30/firefox-3-5-arrives/}}. Siden den tid har de fleste st�rste nettlesere kommet med en generell st�tte, med unntak av den st�rte, Internet Explorer, som vil gi st�tte med neste utgivelse, versjon 9. Noen lanseringsdato er forel�pig ikke fastsatt.

Det er vedvarende ingen enighet om hva slags format som skal brukes p� selve transportstr�mme, hverken innkapsling, eller formater innad i innkapslingen. Safari og den kommende utgaven av Internet Explorer, st�tter utelukkende innkapslingsformatet MP4, med h.264 video- og AAC som lydkodek. Firefox og Opera st�tter kun Ogg som innkapslingsformat, med Theora som video- og Vorbis som lydkodek. Chrome p� den andre siden st�tter begge formater.

I tillegg til utbredelsen av Ogg- og MP4-innkapslingsformatet, er det kommet et tredje alternativ til de to eksistende formatene. Formatet er utviklet av Google og g�r under navnet WebM\footnote{\url{http://www.webmproject.org/}}. Formatet baserer seg p� VP8 som video- og Vorbis som lydkodek, hvor disse kodekene er faste. WebM er tilgjengeliggjort under en �pen lisens som tillater fritt kommerielt bruk. En mer grundig gjennomgang av WebM gj�res i et underkapittel senere i rapporten.\\\\
%
%
\begin{tabular}{| c | l | l | l | l | l | l | l |}
  \hline
  \textbf{Format} & \textbf{IE} & \textbf{Firefox} & \textbf{Safari} & \textbf{Chrome} & \textbf{Opera} & \textbf{iPhone} & \textbf{Android} \\ \hline
  HTML5 video & \parbox[c]{1em}{\includegraphics[width=3mm]{img/icon-ball-blue.png}}\scriptsize{v9.0+} & \parbox[c]{1em}{\includegraphics[width=3mm]{img/icon-ball-green.png}}\scriptsize{v3.5+} & \parbox[c]{1em}{\includegraphics[width=3mm]{img/icon-ball-green.png}}\scriptsize{v3.0+} & \parbox[c]{1em}{\includegraphics[width=3mm]{img/icon-ball-green.png}} & \parbox[c]{1em}{\includegraphics[width=3mm]{img/icon-ball-green.png}} & \parbox[c]{1em}{\includegraphics[width=3mm]{img/icon-ball-green.png}} & \parbox[c]{1em}{\includegraphics[width=3mm]{img/icon-ball-green.png}} \\ \hline

  Ogg\scriptsize{(Theora+Vorbis)}  & \parbox[c]{1em}{\includegraphics[width=3mm]{img/icon-ball-red.png}} & \parbox[c]{1em}{\includegraphics[width=3mm]{img/icon-ball-green.png}}\scriptsize{v3.5+} & \parbox[c]{1em}{\includegraphics[width=3mm]{img/icon-ball-red.png}} & \parbox[c]{1em}{\includegraphics[width=3mm]{img/icon-ball-green.png}}\scriptsize{v5.0+} & \parbox[c]{1em}{\includegraphics[width=3mm]{img/icon-ball-green.png}}\scriptsize{v10.5+} & \parbox[c]{1em}{\includegraphics[width=3mm]{img/icon-ball-red.png}} & \parbox[c]{1em}{\includegraphics[width=3mm]{img/icon-ball-red.png}} \\ \hline

  MP4\scriptsize{(h.262+AAC)}  & \parbox[c]{1em}{\includegraphics[width=3mm]{img/icon-ball-red.png}} & \parbox[c]{1em}{\includegraphics[width=3mm]{img/icon-ball-red.png}} & \parbox[c]{1em}{\includegraphics[width=3mm]{img/icon-ball-green.png}}\scriptsize{v3.0+} & \parbox[c]{1em}{\includegraphics[width=3mm]{img/icon-ball-green.png}}\scriptsize{v5.0+} & \parbox[c]{1em}{\includegraphics[width=3mm]{img/icon-ball-red.png}} & \parbox[c]{1em}{\includegraphics[width=3mm]{img/icon-ball-green.png}}\scriptsize{v3.0+} & \parbox[c]{1em}{\includegraphics[width=3mm]{img/icon-ball-green.png}}\scriptsize{v2.0+} \\ \hline

  WebM\scriptsize{(VP8+Vorbis)}  & \parbox[c]{1em}{\includegraphics[width=3mm]{img/icon-ball-blue.png}}\scriptsize{v9.0+} $^{*}$ & \parbox[c]{1em}{\includegraphics[width=3mm]{img/icon-ball-blue.png}}\scriptsize{v4.0+} & \parbox[c]{1em}{\includegraphics[width=3mm]{img/icon-ball-red.png}} & \parbox[c]{1em}{\includegraphics[width=3mm]{img/icon-ball-blue.png}}\scriptsize{v6.0+} & \parbox[c]{1em}{\includegraphics[width=3mm]{img/icon-ball-blue.png}}\scriptsize{v10.0+} & \parbox[c]{1em}{\includegraphics[width=3mm]{img/icon-ball-red.png}} & \parbox[c]{1em}{\includegraphics[width=3mm]{img/icon-ball-yellow.png}} \\ \hline

\end{tabular}

\scriptsize{* -- brukeren m� installere en plugin}
\footnote{\url{http://news.cnet.com/8301-30685_3-20006239-264.html}}\\\\
%
\begin{tabular}{ c l }
  \parbox[c]{1em}{\includegraphics[width=3mm]{img/icon-ball-green.png}} & st�ttet i siste utgivelse \\
  \parbox[c]{1em}{\includegraphics[width=3mm]{img/icon-ball-blue.png}} & st�tte kommer i gitt fremtidig utgivelse \\
  \parbox[c]{1em}{\includegraphics[width=3mm]{img/icon-ball-yellow.png}} & st�tte kommer i ubestemt fremtidig utgivelse \\
  \parbox[c]{1em}{\includegraphics[width=3mm]{img/icon-ball-red.png}} & ingen st�tte, hverken n� eller i kjent fremtid \\
\end{tabular}

\subsection{WebM}



\subsection{h.264 vs. Ogg vs. WebM}

\subsection{Underteksting}

\subsection{Avspilling i 3.parts program}

\section{Biblioteker}

\section{Publiseringsl�sninger}

\section{�vrig}

\section*{Sjekk ut}
\begin{itemize}
  \item \url{http://hacks.mozilla.org/2010/05/firefox-youtube-and-webm/}
\end{itemize}

\end{document}
